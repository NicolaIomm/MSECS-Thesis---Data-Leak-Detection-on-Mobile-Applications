\chapter{Introduction}
	
	\par Mobile applications are now part of our daily life. Everyone of us make use of different applications during their daily routine. There are applications for literally every kind of necessity: we would like to know what the weather will be like in the next days in our city, we would like to keep track of our fitness training that every morning we do in the neighbourhood, or we would like to know how the traffic is on the way back to our home. Using these applications is becoming a real habit, and like any other habit we do not ask ourselves anymore why or how we are doing it. Going through the years much more often we can hear about \textit{data breaches} or \textit{data leaks} related to applications. The whole amount of data that users has put in their online account can be exposed to risks of this kind. Nowadays people tend to use applications without even realise the amount of informations they are delivering to the network: profile pictures, email addresses, phone numbers. All of these are private informations, and a disclosure of them can be really embarassing for both the user and the company offering the service. \newline
	Mobile applications handle private data of millions of users, therefore it is mandatory checking the robustness and any kind of protection implemented in the application, in order to make it resilient to possible leaks of user informations.\newline
	\par Main topic of this thesis is to investigate multiple mobile applications with a special focus on the personal informations they handle. A critical eye is directed towards any outgoing application request, looking for any vulnerability in the code or in the communication protocol adopted. If by any chance a vulnerability is discovered, then it is mandatory to check if it can be exploited in order to obtain personal informations. \newline
	\par The structure of the thesis proceeds by chapters. This introductive chapter aims to describe the goals of the study and the methodologies used while conducting the investigation on every different mobile application. The second chapter is dedicated to the explaination of the concepts and the technologies met during the work. The third chapter deals with the whole testing environment in which every application has been analyzed, specifying the potentiality of each tool. The fourth chapter explains, application by application, how the investigation has been practically conducted, the data exposed, the vulnerabilities found, and if any, how they can be exploited to obtain private informations. The last chapter is a summary of the whole study, reporting the results of the investigation process in every application.
	
	\subsubsection{Personal Data}
		\par At the really start of this thesis I would like to define the concept of \textbf{Personal Data}. There are multiple definitions and laws defining the scope of this kind of data. As stated by the European Commission~\cite{European Commission}:
		\begin{framed}
			\textit{
				Personal data is any information that relates to an identified or identifiable living individual. Different pieces of information, which collected together can lead to the identification of a particular person, also contribute personal data. \newline
				\indent Personal data that has been de-identified, encrypted or pseudonymised but can be used to re-identify a person remains personal data and falls within the scope of the GDPR. [...] \newline
				\indent The GDPR protects personal data regardless of the technology used for processing that data - it is technology neutral and applies to both automated and manual processing. [...]
			}
		\end{framed}
		\par Practical examples of informations falling in the category of personal data are - name and surname, address, phone number, date of birth, but also photographs, ip addresses, location informations. \newline
		\par Dealing with mobile applications, personal informations are often typed in directly by the user, most likely at the moment of the creation of a new user profile, but not only. Some personal data might be shared automatically during the use of such application, like ip address or advertising phone identifier. \newline
		These information not necessarily are mandatory to access the services offered by the application. Personal data might contribute to some advertising service bundled within the application to keep track of the user data, or these data might be sent to some logging service implemented in the application in order to retrieve statistics on the users utilizing that application. 
	
	\section{Application choice}
		\par The work was conducted by examining applications running on Android operative system. Different mobile applications have been analyzed belonging to three different area of interests:
			\begin{itemize}
				\item \underline{Weather}: General applications \textit{without} user authentication.
				\item \underline{Health \& Fitness}: Applications \textit{with} user authentication and \textit{some} interaction between users.
				\item \underline{Maps \& Navigation}: Sophisticated applications  \textit{with} user authentication and \textit{continuous} interactions with server or other users.
			\end{itemize}
			\par The order in which the applications have been investigated has been decided basing on an increasing complexity and notoriety of them. Starting from weather forecast applications that generally do not require any kind of user authentication, going through fitness applications that let the user customize and share fitness routines to other users, and finally analyzing applications that manage geographical informations to retrieve real time traffic data.
	
	\section{Goals}
		\par Goal of the study in this thesis is to check if software features expose it to the risk of providing more information than planned. While investigating the behaviour of an application there are two questions I have tried to answer. \newline
		The first one is \textit{''Are there any private informations disclosed by the application, and if yes, in which way these informations are protected ?''}. Protecting the customers information is a duty of every company while developing any of its services. It is mandatory ensuring that private and sensitive informations are safely stored. \newline
		After having anwered to that, then the second one \textit{''There is any vulnerability that can be exploited in order to obtain those private informations ?''}. Of course a vulnerability is not present on purpose in an application, the human error is always possible, and it is important be aware of this possibility. \newline
		\par A mobile application might share private informations in different moments of the execution, just while opened, while using it on top of the screen or while running in background. Goal is therefore ensuring there is no leak of private data for the whole activity and at any moment of the application execution. \newline
		 
	\section{Methodologies}
		\par Since the goal is to detect any personal data leak while using the application, dynamic analysis is the main method used to investigate applications behaviour. In particular each mobile application has been inspected starting from the \textbf{network traffic analysis}, generated by the application at runtime. \newline
		Every action carried out by the user in the context of an application will generate some request to a server. Notice that mobile applications do not communicate only with the server offering that service. Indeed in the implementation of a mobile application there are different activities, each one provide a different service - for example advertisement services, logging services, push notification services, and so on. Every request is therefore generated by the application and might potentially include private informations.		
		\par Where the network protocol analysis offered a spotlight for some in-depth study, \textbf{static analysis} has been adopted going through an examination of the code of the application. Performing static analysis on Android applications is really dispersive and time consuming. An Android Package file (APK) can contain hundreds of thousands of Java classes. The majority of the compiled applications is stripped from the symbols, so methods and classes names cannot be directly found in the source code. Most of the time a simple obfuscation method just like the described one, combined with the high number of classes, it is sufficient to hide something very interesting, as the method used to compute the \textit{Authorization header} in an HTTP request is.
		\par To know more about the static and dynamic analysis tools I used in the practice, go to Chapter \ref{chap:testing_environment}.
								
	\section{Collaboration with IPS}
		\par The whole study described in this thesis, has been supported by the company \textit{IPS}\cite{ips}.\newline
		\par IPS was founded in 2000 as a company specialized in the development of solutions and provision of services for the Cyber Security sector and is today an important national reality which offers solutions and products with proprietary technology in the fields of Communication Security and Electronic Surveillance. \newline
		\par The collaboration with IPS and the technological exchange that took place during the research followed two principles:
		\begin{itemize}
			\item \textbf{Cyber defense}: Computer security is a fundamental asset for the business of any company and to defend it is necessary to set up an efficient cyber security system. IPS is a global provider of Cyber Intelligence solutions with more than 30 years of experience in the high-tech market.\newline
			IPS continually revises and refines verification procedures of industrial assets and infrastructures as well as tools used by staff; both for its own protection and as a service offered to customers.\newline
			The continuous analysis of apps, software and vulnerabilities, as done in the specific work of this thesis, is thus part of the core business of IPS.\newline
			\item \textbf{Investigation}: As security is becoming a major concern with threats coming from both the real world and the virtual Internet world, Law Enforcement Agencies see the complexity of investigations increasing and ask for more and more sophisticated technology to provide up to date monitoring capabilities for surveillance of both the real world, the traditional communication services and the Internet. IPS provides LEAs with strategic and tactical solutions for Communication Intelligence needs. Deep knowledge of leaks and data exposure, as done in the specific work of this thesis, are the basis of a technologically advanced investigation work capable of bringing results in the most complex situations.
		\end{itemize}

